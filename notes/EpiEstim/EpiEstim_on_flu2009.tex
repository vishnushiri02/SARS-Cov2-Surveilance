% Options for packages loaded elsewhere
\PassOptionsToPackage{unicode}{hyperref}
\PassOptionsToPackage{hyphens}{url}
%
\documentclass[
]{article}
\usepackage{amsmath,amssymb}
\usepackage{iftex}
\ifPDFTeX
  \usepackage[T1]{fontenc}
  \usepackage[utf8]{inputenc}
  \usepackage{textcomp} % provide euro and other symbols
\else % if luatex or xetex
  \usepackage{unicode-math} % this also loads fontspec
  \defaultfontfeatures{Scale=MatchLowercase}
  \defaultfontfeatures[\rmfamily]{Ligatures=TeX,Scale=1}
\fi
\usepackage{lmodern}
\ifPDFTeX\else
  % xetex/luatex font selection
\fi
% Use upquote if available, for straight quotes in verbatim environments
\IfFileExists{upquote.sty}{\usepackage{upquote}}{}
\IfFileExists{microtype.sty}{% use microtype if available
  \usepackage[]{microtype}
  \UseMicrotypeSet[protrusion]{basicmath} % disable protrusion for tt fonts
}{}
\makeatletter
\@ifundefined{KOMAClassName}{% if non-KOMA class
  \IfFileExists{parskip.sty}{%
    \usepackage{parskip}
  }{% else
    \setlength{\parindent}{0pt}
    \setlength{\parskip}{6pt plus 2pt minus 1pt}}
}{% if KOMA class
  \KOMAoptions{parskip=half}}
\makeatother
\usepackage{xcolor}
\usepackage[margin=1in]{geometry}
\usepackage{color}
\usepackage{fancyvrb}
\newcommand{\VerbBar}{|}
\newcommand{\VERB}{\Verb[commandchars=\\\{\}]}
\DefineVerbatimEnvironment{Highlighting}{Verbatim}{commandchars=\\\{\}}
% Add ',fontsize=\small' for more characters per line
\usepackage{framed}
\definecolor{shadecolor}{RGB}{248,248,248}
\newenvironment{Shaded}{\begin{snugshade}}{\end{snugshade}}
\newcommand{\AlertTok}[1]{\textcolor[rgb]{0.94,0.16,0.16}{#1}}
\newcommand{\AnnotationTok}[1]{\textcolor[rgb]{0.56,0.35,0.01}{\textbf{\textit{#1}}}}
\newcommand{\AttributeTok}[1]{\textcolor[rgb]{0.13,0.29,0.53}{#1}}
\newcommand{\BaseNTok}[1]{\textcolor[rgb]{0.00,0.00,0.81}{#1}}
\newcommand{\BuiltInTok}[1]{#1}
\newcommand{\CharTok}[1]{\textcolor[rgb]{0.31,0.60,0.02}{#1}}
\newcommand{\CommentTok}[1]{\textcolor[rgb]{0.56,0.35,0.01}{\textit{#1}}}
\newcommand{\CommentVarTok}[1]{\textcolor[rgb]{0.56,0.35,0.01}{\textbf{\textit{#1}}}}
\newcommand{\ConstantTok}[1]{\textcolor[rgb]{0.56,0.35,0.01}{#1}}
\newcommand{\ControlFlowTok}[1]{\textcolor[rgb]{0.13,0.29,0.53}{\textbf{#1}}}
\newcommand{\DataTypeTok}[1]{\textcolor[rgb]{0.13,0.29,0.53}{#1}}
\newcommand{\DecValTok}[1]{\textcolor[rgb]{0.00,0.00,0.81}{#1}}
\newcommand{\DocumentationTok}[1]{\textcolor[rgb]{0.56,0.35,0.01}{\textbf{\textit{#1}}}}
\newcommand{\ErrorTok}[1]{\textcolor[rgb]{0.64,0.00,0.00}{\textbf{#1}}}
\newcommand{\ExtensionTok}[1]{#1}
\newcommand{\FloatTok}[1]{\textcolor[rgb]{0.00,0.00,0.81}{#1}}
\newcommand{\FunctionTok}[1]{\textcolor[rgb]{0.13,0.29,0.53}{\textbf{#1}}}
\newcommand{\ImportTok}[1]{#1}
\newcommand{\InformationTok}[1]{\textcolor[rgb]{0.56,0.35,0.01}{\textbf{\textit{#1}}}}
\newcommand{\KeywordTok}[1]{\textcolor[rgb]{0.13,0.29,0.53}{\textbf{#1}}}
\newcommand{\NormalTok}[1]{#1}
\newcommand{\OperatorTok}[1]{\textcolor[rgb]{0.81,0.36,0.00}{\textbf{#1}}}
\newcommand{\OtherTok}[1]{\textcolor[rgb]{0.56,0.35,0.01}{#1}}
\newcommand{\PreprocessorTok}[1]{\textcolor[rgb]{0.56,0.35,0.01}{\textit{#1}}}
\newcommand{\RegionMarkerTok}[1]{#1}
\newcommand{\SpecialCharTok}[1]{\textcolor[rgb]{0.81,0.36,0.00}{\textbf{#1}}}
\newcommand{\SpecialStringTok}[1]{\textcolor[rgb]{0.31,0.60,0.02}{#1}}
\newcommand{\StringTok}[1]{\textcolor[rgb]{0.31,0.60,0.02}{#1}}
\newcommand{\VariableTok}[1]{\textcolor[rgb]{0.00,0.00,0.00}{#1}}
\newcommand{\VerbatimStringTok}[1]{\textcolor[rgb]{0.31,0.60,0.02}{#1}}
\newcommand{\WarningTok}[1]{\textcolor[rgb]{0.56,0.35,0.01}{\textbf{\textit{#1}}}}
\usepackage{graphicx}
\makeatletter
\def\maxwidth{\ifdim\Gin@nat@width>\linewidth\linewidth\else\Gin@nat@width\fi}
\def\maxheight{\ifdim\Gin@nat@height>\textheight\textheight\else\Gin@nat@height\fi}
\makeatother
% Scale images if necessary, so that they will not overflow the page
% margins by default, and it is still possible to overwrite the defaults
% using explicit options in \includegraphics[width, height, ...]{}
\setkeys{Gin}{width=\maxwidth,height=\maxheight,keepaspectratio}
% Set default figure placement to htbp
\makeatletter
\def\fps@figure{htbp}
\makeatother
\setlength{\emergencystretch}{3em} % prevent overfull lines
\providecommand{\tightlist}{%
  \setlength{\itemsep}{0pt}\setlength{\parskip}{0pt}}
\setcounter{secnumdepth}{-\maxdimen} % remove section numbering
\ifLuaTeX
  \usepackage{selnolig}  % disable illegal ligatures
\fi
\IfFileExists{bookmark.sty}{\usepackage{bookmark}}{\usepackage{hyperref}}
\IfFileExists{xurl.sty}{\usepackage{xurl}}{} % add URL line breaks if available
\urlstyle{same}
\hypersetup{
  pdftitle={Epi\_Estim\_on\_flu\_2009},
  hidelinks,
  pdfcreator={LaTeX via pandoc}}

\title{Epi\_Estim\_on\_flu\_2009}
\author{}
\date{\vspace{-2.5em}}

\begin{document}
\maketitle

This is an \href{http://rmarkdown.rstudio.com}{R Markdown} Notebook.
When you execute code within the notebook, the results appear beneath
the code.

\emph{Run} - \emph{Cmd+Shift+Enter}.

\emph{The dataset contains:}

\begin{enumerate}
\def\labelenumi{\arabic{enumi}.}
\tightlist
\item
  The \emph{daily incidence} of onset of acute respiratory illness (ARI,
  defined as at least two symptoms among fever, cough, sore throat, and
  runny nose) amongst children in a school in Pennsylvania during the
  2009 H1N1 influenza pandemic (see Cauchemez et al., PNAS, 2011),
\end{enumerate}

2.The discrete \emph{daily distribution of the serial interval} (time
interval between symptoms onset in a case and in their infector) for
influenza, assuming a shifted Gamma distribution with mean 2.6 days,
standard deviation 1.5 days and shift 1 day (as in Ferguson et al.,
Nature, 2005).

3.\emph{Interval-censored serial interval data} from the 2009 outbreak
of H1N1 influenza in San Antonio, Texas, USA (from Morgan et al., EID
2010).

\emph{Loading the input rda file} which has data regarding incidence,
serial interval and serial interval distribution.

\begin{Shaded}
\begin{Highlighting}[]
\FunctionTok{library}\NormalTok{(EpiEstim)}
\FunctionTok{library}\NormalTok{(ggplot2)}
\FunctionTok{load}\NormalTok{(}\AttributeTok{file=}\StringTok{\textquotesingle{}Flu2009.rda\textquotesingle{}}\NormalTok{)}
\end{Highlighting}
\end{Shaded}

\emph{Data viewing}

\begin{Shaded}
\begin{Highlighting}[]
\DocumentationTok{\#\#incidence}
\NormalTok{Flu2009}\SpecialCharTok{$}\NormalTok{incidence}
\end{Highlighting}
\end{Shaded}

\begin{verbatim}
##         dates  I
## 1  2009-04-27  1
## 2  2009-04-28  1
## 3  2009-04-29  0
## 4  2009-04-30  2
## 5  2009-05-01  5
## 6  2009-05-02  3
## 7  2009-05-03  3
## 8  2009-05-04  3
## 9  2009-05-05  6
## 10 2009-05-06  2
## 11 2009-05-07  5
## 12 2009-05-08  9
## 13 2009-05-09 13
## 14 2009-05-10 12
## 15 2009-05-11 13
## 16 2009-05-12 11
## 17 2009-05-13 12
## 18 2009-05-14  6
## 19 2009-05-15  6
## 20 2009-05-16  6
## 21 2009-05-17  3
## 22 2009-05-18  1
## 23 2009-05-19  0
## 24 2009-05-20  2
## 25 2009-05-21  0
## 26 2009-05-22  0
## 27 2009-05-23  0
## 28 2009-05-24  0
## 29 2009-05-25  2
## 30 2009-05-26  0
## 31 2009-05-27  2
## 32 2009-05-28  0
\end{verbatim}

\begin{Shaded}
\begin{Highlighting}[]
\DocumentationTok{\#\#Serial interval distribution}
\NormalTok{Flu2009}\SpecialCharTok{$}\NormalTok{si\_distr}
\end{Highlighting}
\end{Shaded}

\begin{verbatim}
##  [1] 0.000 0.233 0.359 0.198 0.103 0.053 0.027 0.014 0.007 0.003 0.002 0.001
\end{verbatim}

\begin{Shaded}
\begin{Highlighting}[]
\DocumentationTok{\#\# serial interval data}
\DocumentationTok{\#\# EL/ER show the lower/upper bound of the symptoms onset date in the infector}
\DocumentationTok{\#\# SL/SR show the same for the secondary case}
\DocumentationTok{\#\# type has entries 0 corresponding to doubly interval{-}censored data}
\NormalTok{Flu2009}\SpecialCharTok{$}\NormalTok{si\_data}
\end{Highlighting}
\end{Shaded}

\begin{verbatim}
##    EL ER SL SR type
## 1   0  1  7  8    0
## 2   0  1  2  3    0
## 3   0  1  3  4    0
## 4   0  1  2  5    0
## 5   0  1  1  9    0
## 6   0  1  2  4    0
## 7   0  1  4  6    0
## 8   0  1  4  5    0
## 9   0  1  1  2    0
## 10  0  1  4  5    0
## 11  0  1  5  6    0
## 12  0  1  4  5    0
## 13  0  1  8  9    0
## 14  0  1  3  4    0
## 15  0  1  5  6    0
## 16  0  1  4  5    0
\end{verbatim}

\emph{Plotting daily incidence data against dates using R package called
the incidence.} This package provides functions and classes to compute,
handle and visualise incidence from dated events for a defined time
interval. Dates can be provided in various standard formats. The class
`incidence' is used to store computed incidence and can be easily
manipulated, subsetted, and plotted. In addition, log-linear models can
be fitted to `incidence' objects using `fit'. as.data.frame(object
obtained from as.incidence or incidence function)

\begin{Shaded}
\begin{Highlighting}[]
\FunctionTok{library}\NormalTok{(incidence)}
\FunctionTok{plot}\NormalTok{(}\FunctionTok{as.incidence}\NormalTok{(Flu2009}\SpecialCharTok{$}\NormalTok{incidence}\SpecialCharTok{$}\NormalTok{I, }\AttributeTok{dates =}\NormalTok{ Flu2009}\SpecialCharTok{$}\NormalTok{incidence}\SpecialCharTok{$}\NormalTok{dates))}
\end{Highlighting}
\end{Shaded}

\begin{verbatim}
## Warning: The `guide` argument in `scale_*()` cannot be `FALSE`. This was deprecated in
## ggplot2 3.3.4.
## i Please use "none" instead.
## i The deprecated feature was likely used in the incidence package.
##   Please report the issue at <https://github.com/reconhub/incidence/issues>.
## This warning is displayed once every 8 hours.
## Call `lifecycle::last_lifecycle_warnings()` to see where this warning was
## generated.
\end{verbatim}

\includegraphics{EpiEstim_on_flu2009_files/figure-latex/unnamed-chunk-3-1.pdf}
\#\# Estimating R with parametric serial interval distribution:

\emph{Using estimate\_R on the incidence data to estimate the
reproduction number R}. The time windows over which R is estimated and
the serial interval distribution is specified to this function.the
default behavior is to estimate R over weekly sliding windows. This can
be changed through the config\$t\_start and config\$t\_end arguments
(see below, ``Changing the time windows for estimation''). method
parametric\_si enables just specifiying mean and standard deviation.
When using the parametric\_si an offset gamma distribution is used for
the serial interval.

\begin{Shaded}
\begin{Highlighting}[]
\NormalTok{r\_parametric\_si}\OtherTok{\textless{}{-}}\FunctionTok{estimate\_R}\NormalTok{(Flu2009}\SpecialCharTok{$}\NormalTok{incidence,}\AttributeTok{method=}\StringTok{"parametric\_si"}\NormalTok{,}\AttributeTok{config =} \FunctionTok{make\_config}\NormalTok{(}\FunctionTok{list}\NormalTok{(}\AttributeTok{mean\_si=}\FloatTok{2.6}\NormalTok{,}\AttributeTok{std\_si=}\FloatTok{1.5}\NormalTok{)))}
\end{Highlighting}
\end{Shaded}

\begin{verbatim}
## Default config will estimate R on weekly sliding windows.
##     To change this change the t_start and t_end arguments.
\end{verbatim}

\begin{Shaded}
\begin{Highlighting}[]
\NormalTok{r\_parametric\_si}
\end{Highlighting}
\end{Shaded}

\begin{verbatim}
## $R
##    t_start t_end   Mean(R)     Std(R) Quantile.0.025(R) Quantile.0.05(R)
## 1        2     8 1.7357977 0.40913143        1.02874370       1.12193325
## 2        3     9 1.7491678 0.36472669        1.10882231       1.19547993
## 3        4    10 1.5370581 0.30741163        0.99470299       1.06869352
## 4        5    11 1.4318389 0.27059213        0.95144658       1.01766106
## 5        6    12 1.4227247 0.25150457        0.97314263       1.03580815
## 6        7    13 1.6353733 0.25234358        1.17863324       1.24358968
## 7        8    14 1.6639845 0.23300443        1.23894574       1.30015068
## 8        9    15 1.6161467 0.20692639        1.23622420       1.29149686
## 9       10    16 1.4344239 0.17656537        1.10938331       1.15686977
## 10      11    17 1.4087217 0.16159146        1.10991245       1.15387452
## 11      12    18 1.2479472 0.14221678        0.98486038       1.02359091
## 12      13    19 1.0821477 0.12579711        0.84971868       0.88387071
## 13      14    20 0.9371671 0.11449319        0.72629165       0.75712322
## 14      15    21 0.8196564 0.10762616        0.62239946       0.65101748
## 15      16    22 0.6910352 0.10188759        0.50592464       0.53240938
## 16      17    23 0.5896765 0.09967353        0.41073132       0.43584912
## 17      18    24 0.4992206 0.09984413        0.32306927       0.34710064
## 18      19    25 0.4729631 0.10850519        0.28475471       0.30971499
## 19      20    26 0.4176319 0.11583024        0.22237137       0.24703176
## 20      21    27 0.2978682 0.11258359        0.11975846       0.13979872
## 21      22    28 0.2400880 0.12004401        0.06541591       0.08200917
## 22      23    29 0.4573477 0.20453212        0.14849956       0.18020868
## 23      24    30 0.6734382 0.30117074        0.21866356       0.26535481
## 24      25    31 0.8591897 0.38424130        0.27897655       0.33854643
## 25      26    32 0.9975352 0.44611130        0.32389696       0.39305871
##    Quantile.0.25(R) Median(R) Quantile.0.75(R) Quantile.0.95(R)
## 1         1.4451981 1.7037612        1.9915200        2.4589724
## 2         1.4913446 1.7238839        1.9794569        2.3891206
## 3         1.3200896 1.5166133        1.7317605        2.0751763
## 4         1.2412511 1.4148298        1.6039005        1.9040473
## 5         1.2459958 1.4079324        1.5833401        1.8601072
## 6         1.4588473 1.6224126        1.7977785        2.0713724
## 7         1.5014679 1.6531215        1.8146646        2.0648767
## 8         1.4721862 1.6073240        1.7504931        1.9708944
## 9         1.3117168 1.4271858        1.5492434        1.7366694
## 10        1.2966267 1.4025479        1.5140885        1.6846295
## 11        1.1493086 1.2425490        1.3407028        1.4907183
## 12        0.9948535 1.0772771        1.1641340        1.2970399
## 13        0.8576141 0.9325088        1.0116437        1.1331024
## 14        0.7447274 0.8149506        0.8894575        1.0043487
## 15        0.6198605 0.6860342        0.7567610        0.8667218
## 16        0.5197437 0.5840702        0.6535019        0.7626306
## 17        0.4287515 0.4925804        0.5624580        0.6739958
## 18        0.3959768 0.4646919        0.5409442        0.6644328
## 19        0.3348031 0.4069735        0.4888632        0.6246028
## 20        0.2162803 0.2838104        0.3641850        0.5039247
## 21        0.1521750 0.2204045        0.3066781        0.4653900
## 22        0.3081243 0.4272459        0.5739193        0.8372682
## 23        0.4537089 0.6291137        0.8450883        1.2328660
## 24        0.5788533 0.8026393        1.0781852        1.5729218
## 25        0.6720595 0.9318792        1.2517931        1.8261915
##    Quantile.0.975(R)
## 1          2.6247813
## 2          2.5331192
## 3          2.1955399
## 4          2.0088487
## 5          1.9563365
## 6          2.1657455
## 7          2.1507413
## 8          2.0461979
## 9          1.8005896
## 10         1.7426100
## 11         1.5417063
## 12         1.3422512
## 13         1.1745108
## 14         1.0436503
## 15         0.9045582
## 16         0.8004706
## 17         0.7130887
## 18         0.7081443
## 19         0.6734020
## 20         0.5557145
## 21         0.5262293
## 22         0.9367934
## 23         1.3794155
## 24         1.7598935
## 25         2.0432690
## 
## $method
## [1] "parametric_si"
## 
## $si_distr
##           t0           t1           t2           t3           t4           t5 
## 0.000000e+00 2.331721e-01 3.585794e-01 1.981108e-01 1.033427e-01 5.290518e-02 
##           t6           t7           t8           t9          t10          t11 
## 2.682146e-02 1.351620e-02 6.783438e-03 3.394351e-03 1.694674e-03 8.445945e-04 
##          t12          t13          t14          t15          t16          t17 
## 4.203313e-04 2.089422e-04 1.037609e-04 5.148480e-05 2.552773e-05 1.264952e-05 
##          t18          t19          t20          t21          t22          t23 
## 6.264667e-06 3.101068e-06 1.534394e-06 7.589168e-07 3.752319e-07 1.854675e-07 
##          t24          t25          t26          t27          t28          t29 
## 9.164543e-08 4.527299e-08 2.235952e-08 1.104051e-08 5.450375e-09 2.690170e-09 
##          t30          t31          t32 
## 1.327568e-09 6.550364e-10 0.000000e+00 
## 
## $SI.Moments
##   Mean     Std
## 1  2.6 1.55488
## 
## $dates
##  [1] "2009-04-27" "2009-04-28" "2009-04-29" "2009-04-30" "2009-05-01"
##  [6] "2009-05-02" "2009-05-03" "2009-05-04" "2009-05-05" "2009-05-06"
## [11] "2009-05-07" "2009-05-08" "2009-05-09" "2009-05-10" "2009-05-11"
## [16] "2009-05-12" "2009-05-13" "2009-05-14" "2009-05-15" "2009-05-16"
## [21] "2009-05-17" "2009-05-18" "2009-05-19" "2009-05-20" "2009-05-21"
## [26] "2009-05-22" "2009-05-23" "2009-05-24" "2009-05-25" "2009-05-26"
## [31] "2009-05-27" "2009-05-28"
## 
## $I
##  [1]  1  1  0  2  5  3  3  3  6  2  5  9 13 12 13 11 12  6  6  6  3  1  0  2  0
## [26]  0  0  0  2  0  2  0
## 
## $I_local
##  [1]  0  1  0  2  5  3  3  3  6  2  5  9 13 12 13 11 12  6  6  6  3  1  0  2  0
## [26]  0  0  0  2  0  2  0
## 
## $I_imported
##  [1] 1 0 0 0 0 0 0 0 0 0 0 0 0 0 0 0 0 0 0 0 0 0 0 0 0 0 0 0 0 0 0 0
## 
## attr(,"class")
## [1] "estimate_R"
\end{verbatim}

The function estimate\_R results in Reproductive number for each sliding
window the mean,std,quantiles of R. Apart from that it also gives the
daily si\_distribution, Incidence counts on each day and differentiate
local and imported incidence. I assume the first index case is by
default assumed as imported case. The results are plotted in different
graphs - one for R, one for SI\_distribution and one more for
Incidence(Epidemic curve)

\begin{Shaded}
\begin{Highlighting}[]
\FunctionTok{plot}\NormalTok{(r\_parametric\_si,}\AttributeTok{legend =} \ConstantTok{FALSE}\NormalTok{)}
\end{Highlighting}
\end{Shaded}

\includegraphics{EpiEstim_on_flu2009_files/figure-latex/unnamed-chunk-6-1.pdf}
\#\# Estimating R with non-parametric serial interval distribution:

This is done when the complete distribution of serial interval is known.

\begin{Shaded}
\begin{Highlighting}[]
\NormalTok{r\_non\_parametric\_si }\OtherTok{\textless{}{-}}\FunctionTok{estimate\_R}\NormalTok{(Flu2009}\SpecialCharTok{$}\NormalTok{incidence,}\AttributeTok{method =} \StringTok{"non\_parametric\_si"}\NormalTok{, }\AttributeTok{config =} \FunctionTok{make\_config}\NormalTok{(}\FunctionTok{list}\NormalTok{(}\AttributeTok{si\_distr=}\NormalTok{Flu2009}\SpecialCharTok{$}\NormalTok{si\_distr)))}
\end{Highlighting}
\end{Shaded}

\begin{verbatim}
## Default config will estimate R on weekly sliding windows.
##     To change this change the t_start and t_end arguments.
\end{verbatim}

\emph{Visualization of the estimation}

\begin{Shaded}
\begin{Highlighting}[]
\FunctionTok{plot}\NormalTok{(r\_non\_parametric\_si,}\StringTok{"all"}\NormalTok{)}
\end{Highlighting}
\end{Shaded}

\includegraphics{EpiEstim_on_flu2009_files/figure-latex/unnamed-chunk-8-1.pdf}
\#\# Estimating R accounting for uncertainity of serial interval
distribution: During the early time periods in the outbreak serial
interval distribution is poorly documented. In such cases estimate\_R
allows inclusion of n serial distributions - n pairs of mean and
standard deviation.

\begin{Shaded}
\begin{Highlighting}[]
\NormalTok{config }\OtherTok{\textless{}{-}} \FunctionTok{make\_config}\NormalTok{(}\FunctionTok{list}\NormalTok{(}\AttributeTok{mean\_si =} \FloatTok{2.6}\NormalTok{, }\AttributeTok{std\_mean\_si =} \DecValTok{1}\NormalTok{,}
                           \AttributeTok{min\_mean\_si =} \DecValTok{1}\NormalTok{, }\AttributeTok{max\_mean\_si =} \FloatTok{4.2}\NormalTok{,}
                           \AttributeTok{std\_si =} \FloatTok{1.5}\NormalTok{, }\AttributeTok{std\_std\_si =} \FloatTok{0.5}\NormalTok{,}
                           \AttributeTok{min\_std\_si =} \FloatTok{0.5}\NormalTok{, }\AttributeTok{max\_std\_si =} \FloatTok{2.5}\NormalTok{))}
\NormalTok{r\_uncertain\_si}\OtherTok{\textless{}{-}} \FunctionTok{estimate\_R}\NormalTok{(Flu2009}\SpecialCharTok{$}\NormalTok{incidence, }\AttributeTok{method =} \StringTok{"uncertain\_si"}\NormalTok{, }\AttributeTok{config =}\NormalTok{ config)}
\end{Highlighting}
\end{Shaded}

\begin{verbatim}
## Default config will estimate R on weekly sliding windows.
##     To change this change the t_start and t_end arguments.
\end{verbatim}

\begin{Shaded}
\begin{Highlighting}[]
\FunctionTok{plot}\NormalTok{(r\_uncertain\_si)}
\end{Highlighting}
\end{Shaded}

\includegraphics{EpiEstim_on_flu2009_files/figure-latex/unnamed-chunk-9-1.pdf}
The third plot shows all the Serial interval distributions considered.

\hypertarget{estimating-r-and-the-serial-interval-using-data-on-pairs-infectorinfected}{%
\subsection{Estimating R and the serial interval using data on pairs
infector/infected}\label{estimating-r-and-the-serial-interval-using-data-on-pairs-infectorinfected}}

Serial interval is estimated from the interval censored data using MCMC.
The reproduction number is then estimated using the posterior
distribution of the SI, hence accounting for the uncertainty associated
with this estimate. As the Epidemic progresses newly accounted data can
be incorporated in this method while it is not possible if we use the
parametric\_si method.

\begin{Shaded}
\begin{Highlighting}[]
\NormalTok{MCMC\_seed }\OtherTok{\textless{}{-}} \DecValTok{1}
\NormalTok{overall\_seed }\OtherTok{\textless{}{-}} \DecValTok{2}
\NormalTok{mcmc\_control }\OtherTok{\textless{}{-}} \FunctionTok{make\_mcmc\_control}\NormalTok{(}\AttributeTok{seed =}\NormalTok{ MCMC\_seed, }
                                  \AttributeTok{burnin =} \DecValTok{1000}\NormalTok{)}
\NormalTok{dist }\OtherTok{\textless{}{-}} \StringTok{"G"} \CommentTok{\# fitting a Gamma dsitribution for the SI}
\NormalTok{config }\OtherTok{\textless{}{-}} \FunctionTok{make\_config}\NormalTok{(}\FunctionTok{list}\NormalTok{(}\AttributeTok{si\_parametric\_distr =}\NormalTok{ dist,}
                           \AttributeTok{mcmc\_control =}\NormalTok{ mcmc\_control,}
                           \AttributeTok{seed =}\NormalTok{ overall\_seed, }
                           \AttributeTok{n1 =} \DecValTok{50}\NormalTok{, }
                           \AttributeTok{n2 =} \DecValTok{50}\NormalTok{))}
\NormalTok{res\_si\_from\_data }\OtherTok{\textless{}{-}} \FunctionTok{estimate\_R}\NormalTok{(Flu2009}\SpecialCharTok{$}\NormalTok{incidence,}
                               \AttributeTok{method =} \StringTok{"si\_from\_data"}\NormalTok{,}
                               \AttributeTok{si\_data =}\NormalTok{ Flu2009}\SpecialCharTok{$}\NormalTok{si\_data,}
                               \AttributeTok{config =}\NormalTok{ config)}
\end{Highlighting}
\end{Shaded}

\begin{verbatim}
## Default config will estimate R on weekly sliding windows.
##     To change this change the t_start and t_end arguments.
\end{verbatim}

\begin{verbatim}
## Running 1500 MCMC iterations 
## MCMCmetrop1R iteration 1 of 1500 
## function value =  -27.68336
## theta = 
##    1.67335
##   -0.27928
## Metropolis acceptance rate = 0.00000
## 
## MCMCmetrop1R iteration 1001 of 1500 
## function value =  -29.17093
## theta = 
##    1.07173
##    0.18715
## Metropolis acceptance rate = 0.54246
## 
## 
## 
## @@@@@@@@@@@@@@@@@@@@@@@@@@@@@@@@@@@@@@@@@@@@@@@@@@@@@@@@@
## The Metropolis acceptance rate was 0.55267
## @@@@@@@@@@@@@@@@@@@@@@@@@@@@@@@@@@@@@@@@@@@@@@@@@@@@@@@@@
## 
## Gelman-Rubin MCMC convergence diagnostic was successful.
## 
## @@@@@@@@@@@@@@@@@@@@@@@@@@@@@@@@@@@@@@@@@@@@@@@@@@@@@@@@@ 
## Estimating the reproduction number for these serial interval estimates...
##  @@@@@@@@@@@@@@@@@@@@@@@@@@@@@@@@@@@@@@@@@@@@@@@@@@@@@@@@@
\end{verbatim}

\begin{Shaded}
\begin{Highlighting}[]
\FunctionTok{plot}\NormalTok{(res\_si\_from\_data, }\AttributeTok{legend =} \ConstantTok{FALSE}\NormalTok{)}
\end{Highlighting}
\end{Shaded}

\includegraphics{EpiEstim_on_flu2009_files/figure-latex/unnamed-chunk-10-1.pdf}
\#\# Changing the time window of estimation Changing the time window
through config\(t_start and config\)t\_end First one would look similar
to the previous parametric\_si results because it is weekly

\begin{Shaded}
\begin{Highlighting}[]
\NormalTok{T }\OtherTok{\textless{}{-}} \FunctionTok{nrow}\NormalTok{(Flu2009}\SpecialCharTok{$}\NormalTok{incidence)}
\DocumentationTok{\#\#32}
\NormalTok{t\_start }\OtherTok{\textless{}{-}} \FunctionTok{seq}\NormalTok{(}\DecValTok{2}\NormalTok{, T}\DecValTok{{-}6}\NormalTok{) }\CommentTok{\# starting at 2 as conditional on the past observations}
\DocumentationTok{\#\#2  3  4  5  6  7  8  9 10 11 12 13 14 15 16 17 18 19 20 21 22 23 24 25 26}
\NormalTok{t\_end }\OtherTok{\textless{}{-}}\NormalTok{ t\_start }\SpecialCharTok{+} \DecValTok{6} \CommentTok{\# adding 6 to get 7{-}day windows as bounds included in window}
\DocumentationTok{\#\#8  9 10 11 12 13 14 15 16 17 18 19 20 21 22 23 24 25 26 27 28 29 30 31 32}
\NormalTok{res\_weekly }\OtherTok{\textless{}{-}} \FunctionTok{estimate\_R}\NormalTok{(Flu2009}\SpecialCharTok{$}\NormalTok{incidence, }
                         \AttributeTok{method=}\StringTok{"parametric\_si"}\NormalTok{,}
                         \AttributeTok{config =} \FunctionTok{make\_config}\NormalTok{(}\FunctionTok{list}\NormalTok{(}
                           \AttributeTok{t\_start =}\NormalTok{ t\_start,}
                           \AttributeTok{t\_end =}\NormalTok{ t\_end,}
                           \AttributeTok{mean\_si =} \FloatTok{2.6}\NormalTok{, }
                           \AttributeTok{std\_si =} \FloatTok{1.5}\NormalTok{))}
\NormalTok{)}
\FunctionTok{plot}\NormalTok{(res\_weekly, }\StringTok{"all"}\NormalTok{) }
\end{Highlighting}
\end{Shaded}

\includegraphics{EpiEstim_on_flu2009_files/figure-latex/unnamed-chunk-11-1.pdf}
\emph{Bi weekly}

\begin{Shaded}
\begin{Highlighting}[]
\NormalTok{t\_start }\OtherTok{\textless{}{-}} \FunctionTok{seq}\NormalTok{(}\DecValTok{2}\NormalTok{, T}\DecValTok{{-}13}\NormalTok{) }\CommentTok{\# starting at 2 as conditional on the past observations}
\DocumentationTok{\#\# 2  3  4  5  6  7  8  9 10 11 12 13 14 15 16 17 18 19}
\NormalTok{t\_end }\OtherTok{\textless{}{-}}\NormalTok{ t\_start }\SpecialCharTok{+} \DecValTok{13} 
\DocumentationTok{\#\#15 16 17 18 19 20 21 22 23 24 25 26 27 28 29 30 31 32}
\NormalTok{res\_biweekly }\OtherTok{\textless{}{-}} \FunctionTok{estimate\_R}\NormalTok{(Flu2009}\SpecialCharTok{$}\NormalTok{incidence, }
                           \AttributeTok{method=}\StringTok{"parametric\_si"}\NormalTok{,}
                           \AttributeTok{config =} \FunctionTok{make\_config}\NormalTok{(}\FunctionTok{list}\NormalTok{(}
                             \AttributeTok{t\_start =}\NormalTok{ t\_start,}
                             \AttributeTok{t\_end =}\NormalTok{ t\_end,}
                             \AttributeTok{mean\_si =} \FloatTok{2.6}\NormalTok{, }
                             \AttributeTok{std\_si =} \FloatTok{1.5}\NormalTok{))}
\NormalTok{)}
\FunctionTok{plot}\NormalTok{(res\_biweekly, }\StringTok{"all"}\NormalTok{) }
\end{Highlighting}
\end{Shaded}

\includegraphics{EpiEstim_on_flu2009_files/figure-latex/unnamed-chunk-12-1.pdf}
Estimating R on intervals of interest

\begin{Shaded}
\begin{Highlighting}[]
\NormalTok{t\_start }\OtherTok{\textless{}{-}} \FunctionTok{c}\NormalTok{(}\DecValTok{2}\NormalTok{, }\DecValTok{18}\NormalTok{, }\DecValTok{25}\NormalTok{) }\CommentTok{\# starting at 2 as conditional on the past observations}
\NormalTok{t\_end }\OtherTok{\textless{}{-}} \FunctionTok{c}\NormalTok{(}\DecValTok{17}\NormalTok{, }\DecValTok{24}\NormalTok{, }\DecValTok{32}\NormalTok{)}
\NormalTok{res\_before\_during\_after\_closure }\OtherTok{\textless{}{-}} \FunctionTok{estimate\_R}\NormalTok{(Flu2009}\SpecialCharTok{$}\NormalTok{incidence, }
                                              \AttributeTok{method=}\StringTok{"parametric\_si"}\NormalTok{,}
                                              \AttributeTok{config =} \FunctionTok{make\_config}\NormalTok{(}\FunctionTok{list}\NormalTok{(}
                                                \AttributeTok{t\_start =}\NormalTok{ t\_start,}
                                                \AttributeTok{t\_end =}\NormalTok{ t\_end,}
                                                \AttributeTok{mean\_si =} \FloatTok{2.6}\NormalTok{, }
                                                \AttributeTok{std\_si =} \FloatTok{1.5}\NormalTok{))}
\NormalTok{)}
\FunctionTok{plot}\NormalTok{(res\_before\_during\_after\_closure,}\StringTok{"R"}\NormalTok{) }\SpecialCharTok{+} \FunctionTok{geom\_hline}\NormalTok{(}\FunctionTok{aes}\NormalTok{(}\AttributeTok{yintercept=}\DecValTok{1}\NormalTok{),}\AttributeTok{color=}\StringTok{"red"}\NormalTok{, }\AttributeTok{lty=} \DecValTok{2}\NormalTok{)}
\end{Highlighting}
\end{Shaded}

\includegraphics{EpiEstim_on_flu2009_files/figure-latex/unnamed-chunk-13-1.pdf}
\#\# Different way to specify incidence

Incidence can just be the counts of the cases and not including the
dates. This will just have impact on x axis

\begin{Shaded}
\begin{Highlighting}[]
\NormalTok{config }\OtherTok{\textless{}{-}} \FunctionTok{make\_config}\NormalTok{(}\FunctionTok{list}\NormalTok{(}\AttributeTok{mean\_si =} \FloatTok{2.6}\NormalTok{, }\AttributeTok{std\_si =} \FloatTok{1.5}\NormalTok{))}
\NormalTok{res\_incid\_vector }\OtherTok{\textless{}{-}} \FunctionTok{estimate\_R}\NormalTok{(Flu2009}\SpecialCharTok{$}\NormalTok{incidence}\SpecialCharTok{$}\NormalTok{I, }
                               \AttributeTok{method=}\StringTok{"parametric\_si"}\NormalTok{,}
                               \AttributeTok{config =}\NormalTok{ config)}
\end{Highlighting}
\end{Shaded}

\begin{verbatim}
## Default config will estimate R on weekly sliding windows.
##     To change this change the t_start and t_end arguments.
\end{verbatim}

\begin{Shaded}
\begin{Highlighting}[]
\FunctionTok{plot}\NormalTok{(res\_incid\_vector,}\StringTok{"all"}\NormalTok{)}
\end{Highlighting}
\end{Shaded}

\includegraphics{EpiEstim_on_flu2009_files/figure-latex/unnamed-chunk-14-1.pdf}
\#\# Estimating R just by giving the dates for each case rather than the
count on each dates

The following command manipulates the dates data set - if on a
particular date there are 5 cases, then this date is repeated 5 times.
rep(i,Flu2009\$incidence\$I{[}i{]}) =\textgreater{} i is the iteration
number, Flu2009\(incidence\)I{[}i{]} denotes the ith data in the \$I of
incidence. i is then repeated Flu2009\(incidence\)I{[}i{]} times.

\begin{Shaded}
\begin{Highlighting}[]
\NormalTok{T }\OtherTok{\textless{}{-}} \FunctionTok{nrow}\NormalTok{(Flu2009}\SpecialCharTok{$}\NormalTok{incidence)}
\NormalTok{dates\_onset }\OtherTok{\textless{}{-}}\NormalTok{ Flu2009}\SpecialCharTok{$}\NormalTok{incidence}\SpecialCharTok{$}\NormalTok{dates[}\FunctionTok{unlist}\NormalTok{(}\FunctionTok{lapply}\NormalTok{(}\FunctionTok{seq\_len}\NormalTok{(}\FunctionTok{nrow}\NormalTok{(Flu2009}\SpecialCharTok{$}\NormalTok{incidence)), }\ControlFlowTok{function}\NormalTok{(i) }
  \FunctionTok{rep}\NormalTok{(i, Flu2009}\SpecialCharTok{$}\NormalTok{incidence}\SpecialCharTok{$}\NormalTok{I[i])))]}
\NormalTok{last\_date }\OtherTok{\textless{}{-}}\NormalTok{ Flu2009}\SpecialCharTok{$}\NormalTok{incidence}\SpecialCharTok{$}\NormalTok{date[T]}
\NormalTok{res\_incid\_class }\OtherTok{\textless{}{-}} \FunctionTok{estimate\_R}\NormalTok{(}\FunctionTok{incidence}\NormalTok{(dates\_onset, }\AttributeTok{last\_date =}\NormalTok{ last\_date), }
                              \AttributeTok{method=}\StringTok{"parametric\_si"}\NormalTok{,}
                              \AttributeTok{config =}\NormalTok{ config)}
\end{Highlighting}
\end{Shaded}

\begin{verbatim}
## Default config will estimate R on weekly sliding windows.
##     To change this change the t_start and t_end arguments.
\end{verbatim}

\begin{Shaded}
\begin{Highlighting}[]
\FunctionTok{plot}\NormalTok{(res\_incid\_class,}\StringTok{"all"}\NormalTok{)}
\end{Highlighting}
\end{Shaded}

\includegraphics{EpiEstim_on_flu2009_files/figure-latex/unnamed-chunk-15-1.pdf}
\#\# specifying imported cases

Including the information regarding local or importation of cases.

\begin{Shaded}
\begin{Highlighting}[]
\NormalTok{location }\OtherTok{\textless{}{-}} \FunctionTok{sample}\NormalTok{(}\FunctionTok{c}\NormalTok{(}\StringTok{"local"}\NormalTok{,}\StringTok{"imported"}\NormalTok{), }\FunctionTok{length}\NormalTok{(dates\_onset), }\AttributeTok{replace=}\ConstantTok{TRUE}\NormalTok{) }\CommentTok{\#Randomply assigning each cases on each day as locaal or imported.}
\NormalTok{location[}\DecValTok{1}\NormalTok{]}\OtherTok{\textless{}{-}} \StringTok{"imported"} \CommentTok{\#Forcing the first case to be imported}
\NormalTok{incid}\OtherTok{\textless{}{-}}\FunctionTok{incidence}\NormalTok{(dates\_onset,}\AttributeTok{groups =}\NormalTok{ location)}
\FunctionTok{plot}\NormalTok{(incid)}
\end{Highlighting}
\end{Shaded}

\includegraphics{EpiEstim_on_flu2009_files/figure-latex/unnamed-chunk-16-1.pdf}

\begin{Shaded}
\begin{Highlighting}[]
\NormalTok{res\_with\_imports }\OtherTok{\textless{}{-}} \FunctionTok{estimate\_R}\NormalTok{(incid,}\AttributeTok{method=}\StringTok{"parametric\_si"}\NormalTok{, }\AttributeTok{config =} \FunctionTok{make\_config}\NormalTok{(}\FunctionTok{list}\NormalTok{(}
                   \AttributeTok{mean\_si =} \FloatTok{2.6}\NormalTok{, }\AttributeTok{std\_si =} \FloatTok{1.5}\NormalTok{)))}
\end{Highlighting}
\end{Shaded}

\begin{verbatim}
## Default config will estimate R on weekly sliding windows.
##     To change this change the t_start and t_end arguments.
\end{verbatim}

\begin{Shaded}
\begin{Highlighting}[]
\FunctionTok{plot}\NormalTok{(res\_with\_imports, }\AttributeTok{add\_imported\_cases=}\ConstantTok{TRUE}\NormalTok{) }
\end{Highlighting}
\end{Shaded}

\begin{verbatim}
## The number of colors (8) did not match the number of groups (2).
## Using `col_pal` instead.
\end{verbatim}

\includegraphics{EpiEstim_on_flu2009_files/figure-latex/unnamed-chunk-17-1.pdf}
add\_imported\_cases: A boolean to specify whether, on the incidence
time series plot, to add the incidence of imported cases. Add a new
chunk by clicking the \emph{Insert Chunk} button on the toolbar or by
pressing \emph{Cmd+Option+I}.

When you save the notebook, an HTML file containing the code and output
will be saved alongside it (click the \emph{Preview} button or press
\emph{Cmd+Shift+K} to preview the HTML file).

The preview shows you a rendered HTML copy of the contents of the
editor. Consequently, unlike \emph{Knit}, \emph{Preview} does not run
any R code chunks. Instead, the output of the chunk when it was last run
in the editor is displayed.

\end{document}
